\section{Assembling Paired-end Reads}

\begin{note}
The data you will examine in this exercise is again from Staphylococcus aureus
which has a genome of around 3MB. The reads are Illumina paired end with an
insert size of 350 bp.

The required data can be downloaded from the SRA. Specifically, the run data
(SRR022852) from the SRA Sample SRS004748.

\center{\url{http://www.ebi.ac.uk/ena/data/view/SRS004748}}

\end{note}

\begin{information}
The following exercise focuses on preparing the paired-end FASTQ files ready for
velvet, using velvet in paired-end mode and comparing results with velvet's
'auto' option.
\end{information}

\begin{steps}
First move to the directory you made for this exercise and make a suitable named
directory for the exercise:
\begin{lstlisting}
cd ~/NGS/velvet/part2 
mkdir SRS004748 
cd SRS004748
\end{lstlisting}

There is no need to download1 the read files, as they are already stored
locally. You will simply create a soft link to this pre-downloaded data using
the following commands:
\begin{lstlisting}
ln -s ~/NGS/Data/SRR022852_?.fastq.gz ./
\end{lstlisting}

It would be interesting to monitor the way that the programs you will run
utilize your computer's resources, particularly memory. A simple way to do this
is to open a second terminal and in it type:
\begin{lstlisting}
top
\end{lstlisting}
\end{steps}

\begin{note}
\texttt{top} is a program that continually monitors all the processes running on
your computer, showing the resources used by each. Alternatively you could use a
system monitor from your operating system as well. Leave this running and refer
to it at intervals, especially when programs appear to be taking a long time, of
just whenever your curiosity gets the better of you. You should find that as
this practical progresses, memory usage will increase significantly, but
hopefully not beyond the capacity of your workstation.

Now, back to the first terminal, you are ready to run \texttt{velveth} and \texttt{velvetg}. The
reads are \texttt{-shortPaired} this time and for the first run you should not use any
parameters for \texttt{velvetg}.
\end{note}

\begin{information}
From this point on, where it will be informative, time
your runs. This is very easy to do, just prefix the command to run the program
with the command \texttt{time}. This will cause UNIX to report how long the program took
to complete its task.
\end{information}

\begin{steps}
Set the two stages of velvet running, whilst you watch the memory usage as
reported by \texttt{top}. time the \texttt{velvetg} stage. The commands to enter
are:
\begin{lstlisting}
velveth run_25 25 -fmtAuto -create_binary -shortPaired -separate SRR022852_1.fastq.gz SRR022852_2.fastq.gz
time velvetg run_25
\end{lstlisting}

\end{steps}

\begin{questions}
What does \texttt{-fmtAuto} and \texttt{-create\_binary} do? (see help menu)
\begin{answer}
TODO
\end{answer}

Comment on the use of memory and CPU for \texttt{velveth} and \texttt{velvetg}?
\begin{answer}
\texttt{velveth} uses only one CPU while \texttt{velvetg} uses all possible CPUs for some parts of the calculation.
\end{answer}
 
How long did \texttt{velvetg} take?
\begin{answer}
My own measurements on an EBI machine are:\\
\texttt{real    1m8.877s; user    4m15.324s; sys     0m4.716s}
\end{answer}

\end{questions}

\begin{steps}
Next, after saving your \texttt{contigs.fa} file from being overwritten, set the
cut-off parameters that you investigated in the previous exercise and rerun
\texttt{velvetg}.
time and monitor the use of resources as previously. Start with
\texttt{-cov\_cutoff 16} thus:
\begin{lstlisting}
mv run_25/contigs.fa run_25/contigs.fa.0
time velvetg run_25 -cov_cutoff 16
\end{lstlisting}

Up until now, \texttt{velvetg} has ignored the paired-end information. Now try
running
\texttt{velvetg} with both \texttt{-cov\_cutoff 16} and \texttt{-exp\_cov 26}, but first save your \texttt{contigs.fa}
file. By using \texttt{-cov\_cutoff} and \texttt{-exp\_cov}, \texttt{velvetg}
tries to estimate the insert length, which you will see in the \texttt{velvetg}
output. The command is, of course:
\begin{lstlisting}
mv run_25/contigs.fa run_25/contigs.fa.1
time velvetg run_25 -cov_cutoff 16 -exp_cov 26
\end{lstlisting}

\end{steps}

\begin{questions}
Comment on the time required, use of memory and CPU for \texttt{velvetg}?
\begin{answer}
Runtime is lower when velvet can reuse previously calculated data. By using \texttt{-exp\_cov}, the memory usage increases.
\end{answer}

Which insert length does Velvet estimate?
\begin{answer}
Paired-end library 1 has length: 228, sample standard deviation: 26
\end{answer}

\end{questions}

\begin{steps}
Next try running \texttt{velvetg} in `paired-end mode`. This entails running \texttt{velvetg}
specifying the insert length with the parameter \texttt{-ins\_length} set to 350. Even
though velvet estimates the insert length it is always advisable to check /
provide the insert length manually as velvet can get the statistics wrong due
to noise. Just in case, save your last version of \texttt{contigs.fa}. The commands are:
\begin{lstlisting}
mv run_25/contigs.fa run_25/contigs.fa.2
time velvetg run_25 -cov_cutoff 16 -exp_cov 26 -ins_length 350
mv run_25/contigs.fa run_25/contigs.fa.3
\end{lstlisting}
\end{steps}

\begin{questions}
How fast was this run?
\begin{answer}
Down to my own measurements on an EBI machine:\\
\texttt{real    0m29.792s; user    1m4.372s; sys     0m3.880s}
\end{answer}

\end{questions}

\begin{steps}
Take a look into the Log file.
\end{steps}

\begin{questions}
What is the N50 value for the \texttt{velvetg} runs using the switches:\\
\begin{answer}
Base run: 19,510 bp
\end{answer}
\texttt{-cov\_cutoff 16}
\begin{answer}
24,739 bp
\end{answer}

\texttt{-cov\_cutoff 16 -exp\_cov 26}
\begin{answer}
61,793 bp
\end{answer}

\texttt{-cov\_cutoff 16 -exp\_cov 26 -ins\_length 350}
\begin{answer}
n50 of 62,740 bp; max 194,649 bp; total 2,871,093 bp
\end{answer}

\end{questions}

\begin{steps}
Try giving the \texttt{-cov\_cutoff} and/or \texttt{-exp\_cov} parameters the
value \texttt{auto} - the \texttt{velvetg} help output could show you how. The
information Velvet prints during running includes information about the values
used (coverage cut-off or insert length) when using the \texttt{auto} option.
\end{steps}

\begin{questions}
Which coverage values does Velvet choose (hint: look at the output which velvet
produces while running)?
\begin{answer}
Median coverage depth = 26.021837\\
Removing contigs with coverage < 13.010918 \ldots
\end{answer}

How does the N50 value change?
\begin{answer}
n50 of 68,843 bp; max 194,645 bp; total 2,872,678 bp
\end{answer}
\end{questions}

\begin{steps}
Run \texttt{gnx} on all the \texttt{contig.fa} files you have generated in the
course of this exercise. The command will be:
\begin{lstlisting}
gnx -min 100 -nx 25,50,75 run_25/contigs.fa*
\end{lstlisting}
\end{steps}

\begin{questions}
For which runs are there Ns in the \texttt{contigs.fa} file and why? 
\begin{answer}
contigs.fa.2, contigs.fa.3, contigs.fa\\
Velvet tries to use the provided (or infers) the insert length and fills ambiguous regions with Ns.
\end{answer}

Comment on the number of contigs and total length generated for each run.
\begin{answer}
\begin{table}[H]
  %\centering
    \begin{tabular}{ll}
    \toprule
    Filename & \# contigs & Total length & \# Ns \\
    \midrule
    Contigs.fa.0 & 631 & 2,830,659 & 0 \\
    Contigs.fa.1 & 580 & 2,832,670 & 0 \\
    Contigs.fa.2 & 166 & 2,849,919 & 4,847 \\
    Contigs.fa.3 & 166 & 2,856,795 & 11,713 \\
    Contigs.fa & 163 & 2,857,439 & 11,526 \\
    \bottomrule
    \end{tabular}%
  \label{tab:velvetrunresults}%
\end{table}%
\end{answer}
\end{questions}

\begin{bonus}
Similar to the single-end assembly you created previously, we can output a
paired-end assembly as an AMOS message format file. This file can then be
converted into an AMOS bank and viewed using AMOS Hawkeye.
\begin{lstlisting}
time velvetg run_25 -cov_cutoff 16 -exp_cov 26 -ins_length 350 -amos_file yes -read_trkg yes 
time bank-transact -c -b run_25/velvet_asm.bnk -m run_25/velvet_asm.afg         
hawkeye run_25/velvet_asm.bnk
\end{lstlisting}

\begin{questions}
Looking at the scaffold view of a contig, comment on the proportion of happy
mates to compressed mates.
\begin{answer}
TODO
\end{answer}

What is the mean and standard deviation of the insert size report in the
Libraries tab of Hawkeye?
\begin{answer}
TODO
\end{answer}

Look at the actual distribution of insert sizes for this library. Can you
explain where there is a difference between the mean and SD reported in those
two places?
\begin{answer}
TODO
\end{answer}
\end{questions}

\begin{steps}
You can get AMOS to re-estimate the mean and SD of insert sizes. First, close
Hawkeye and then run the following commands before reopening the AMOS bank to
see what has changed.
\begin{lstlisting}
asmQC -b run_25/velvet_asm.bnk -scaff -recompute -update -numsd 2
hawkeye run_25/velvet_asm.bnk
\end{lstlisting}
\end{steps}

\begin{questions}
Looking at the scaffold view of a contig, comment on the proportion of happy
mates to compressed mates.
\begin{answer}
TODO
\end{answer}

What is the mean and standard deviation of the insert size report in the
Libraries tab of Hawkeye?
\begin{answer}
TODO
\end{answer}

Look at the actual distribution of insert sizes for this library. Does the mean
and SD reported in each of those two places now match?
\begin{answer}
TODO
\end{answer}
\end{questions}
\end{bonus}

\subsection{Data Quality}
\begin{note}
As discussed previously, fastq format files include quality assessment of each
called base. As also mentioned earlier, velvet does not use this information
directly. However, by taking into account the quality of the read data, it
should be possible to both improve the efficiency (in terms of memory usage and
runtime) and the quality of the output.
\end{note}

\begin{information}
To investigate the effect of data quality, we will use the run data (SRR023408)
from the SRA experiment SRX008042. The reads are Illumina paired end with an
insert size of 92 bp.
\end{information}

\begin{steps}
As ever, move up a directory to the main directory for the whole of this
exercise and create and enter a new directory dedicated to this phase of the
exercise. The commands are:
\begin{lstlisting}
cd ~/NGS/velvet/part2 
mkdir SRX008042 
cd SRX008042
\end{lstlisting}

Create soft links to the read data files we downloaded for you from the SRA with
the command:
\begin{lstlisting}
ln -s ~/NGS/Data/SRR023408_?.fastq.gz ./
\end{lstlisting}
\end{steps}

\begin{note}
To look at the quality scores, you could use a tool called FastQC to process and
present the scores in a compact manner of basic statistics. FastQC can be
downloaded from:

\center{\url{http://www.bioinformatics.bbsrc.ac.uk/projects/fastqc/}}

Where there are links to an excellent manual and an even more excellent youtube
video. I hope to show you the video and recommend you keep the manual pages open
whilst you are using FastQC in this exercise.
\end{note}

\begin{steps}
Start up FastQC Graphical User Interface (GUI) to examine your compressed fastq
files by typing:
\begin{lstlisting}
fastqc &
\end{lstlisting}
\end{steps}

\begin{note}
The \texttt{\&} sets FastQC running as a background job, thus freeing your terminal for
further commands if required. FastQC emerges as a big blank window.
\end{note}

\begin{steps}
Open both your compressed fastq files as was done in the video (Choose Open from
the File pull down menu, browse to your files, select them both and click OK).
Look at tabs for both files:
\end{steps}

\begin{figure}[H]
\centering
\includegraphics[width=0.8\textwidth]{de_novo/paired_fastqc.png}
\label{fig:paired_fastqc}
\end{figure}

\begin{questions}
Are the quality scores the same for both files?
\begin{answer}
Overall yes
\end{answer}

Which value varies?
\begin{answer}
Per sequence quality scores
\end{answer}

Take a look at the Per base sequence quality for both files. Did you note that it is not good for either file?
\begin{answer}
The quality score of both files drop very fast, the REV strand faster than the FWD strand
\end{answer}

At which positions would you cut the reads off?
\begin{answer}
Looking at the �Per base quality� and �Per base sequence content, both around position 27
\end{answer}

Why does the quality deteriorate towards the end of the read?
\begin{answer}
Errors more likely for later cycles
\end{answer}

Does it make sense to trim the start?
\begin{answer}
Looking at the �Per base sequence content�, yes - there is a clear signal at the beginning.
\end{answer}
\end{questions}

\begin{steps}
Have a look at the other options that fastqc offers. Make good use of the Help
to remind you of what you learned from the video. I suggest you open the Help
window and keep it on your screen, open at the relevant page, as you browse
through the fastqc possibilities.
\end{steps}

\begin{questions}
Which other statistics could you use to support your trimming strategy?
\begin{answer}
�Per base sequence content�, �Per base GC content�, �Kmer content�, �Per base sequence quality�
\end{answer}
\end{questions}

\begin{figure}[H]
\centering
\includegraphics[width=0.8\textwidth]{de_novo/paired_fastqc_quality_plots.png}
\label{fig:paired_fastqc_quality_plots}
\end{figure}

\begin{steps}
Once you have decided how it might be sensible to trim your reads, close down
fastqc by selecting Exit from the File pull down menu. Now apply the knife to
your reads with \texttt{fastx\_trimmer} from the FASTX-Toolkit. However, you will
first need to decompress the FASTQ files. The usage of the tool can be found
here:

\url{http://hannonlab.cshl.edu/fastx_toolkit/commandline.html#fastx_trimmer_usage}

The suggestion (hopefully not far from your own thoughts?) is that you trim your
reads as follows:

\begin{lstlisting}
gunzip < SRR023408_1.fastq.gz > SRR023408_1.fastq
gunzip < SRR023408_2.fastq.gz > SRR023408_2.fastq
fastx_trimmer -Q 33 -f 4 -l 32 -i SRR023408_1.fastq -o SRR023408_trim1.fastq 
fastx_trimmer -Q 33 -f 3 -l 29 -i SRR023408_2.fastq -o SRR023408_trim2.fastq
\end{lstlisting}

Now run velveth with a k-mer value of 21 for both the untrimmed and trimmed read
files in \texttt{-shortPaired} mode. Separate the output of the two executions
of \texttt{velveth} into suitably named directories, followed by
\texttt{velvetg}. The commands would be:
\begin{lstlisting}
velveth run_21 21 -fmtAuto -create_binary -shortPaired -separate SRR023408_1.fastq SRR023408_2.fastq
time velvetg run_21

velveth run_21trim 21 -fmtAuto -create_binary -shortPaired -separate SRR023408_trim1.fastq SRR023408_trim2.fastq
time velvetg run_21trim
\end{lstlisting}
\end{steps}

\begin{questions}
What are the times for the two \texttt{velvetg} runs?
\begin{answer}
run\_25:      \texttt{real    3m16.132s; user    8m18.261s; sys     0m7.317s}\\
run\_25trim:  \texttt{real    1m18.611s; user    3m53.140s; sys     0m4.962s}
\end{answer}

What N50 scores did you achieve?
\begin{answer}
11 (untrimmed), 115 (trimmed)
\end{answer}

What were the overall effects of trimming?
\begin{answer}
Time saving, increased N50, reduced coverage
\end{answer}
\end{questions}

\begin{steps}
The evidence is that trimming improved the assembly. The thing to do surely, is
to run velvetg with the \texttt{-cov\_cutoff} and \texttt{-exp\_cov}. In order
to use \texttt{-cov\_cutoff} and \texttt{-exp\_cov} sensibly, you need to
investigate with R, as you did in the previous exercise, what parameter values
to use. To start up R and produce the weighted histograms, type:
\begin{lstlisting}
R --no-save
library(plotrix) 
data <- read.table("run_21/stats.txt", header=TRUE) 
data2 <- read.table("run_21trim/stats.txt", header=TRUE) 
x11()
par(mfrow=c(1,2))
weighted.hist(data$short1_cov, data$lgth, breaks=0:50)
weighted.hist(data2$short1_cov, data2$lgth, breaks=0:50)
\end{lstlisting}

\begin{figure}[H]
\centering
\includegraphics[width=0.8\textwidth]{de_novo/velvet_Rplot001.png}
\label{fig:velvet_Rplot001}
\end{figure}

One histogram suggests to me an expected coverage of around 13 with a coverage
cut-off of around 7. The trimmed histogram instead suggests an expected coverage
of around 9 with a coverage cut-off of around 5. If you disagree, feel free to
try different settings, but first leave R before running velvetg:
\begin{lstlisting}
q()

time velvetg run_21 -cov_cutoff 7 -exp_cov 13 -ins_length 92
time velvetg run_21trim -cov_cutoff 5 -exp_cov 9 -ins_length 92
\end{lstlisting}

\end{steps}

\begin{questions}
How good does it look now?\\
\begin{answer}
Still not great
\end{answer}
Comment on:\\ 
Runtime
\begin{answer}
Reduced runtime
\end{answer}

Memory
\begin{answer}
Lower memory usage
\end{answer}

k-mer choice (Can you use kmer 31 for a read of length 30 bp?)
\begin{answer}
K-mer has to be lower than the read length and the Kmer coverage should be sufficient to produce results.
\end{answer}

Does less data mean 'worse' results?
\begin{answer}
Not necessarily. If you have lots of data you can savely remove poor data without affecting overall coverage.
\end{answer}
  
How would a smaller/larger Kmer size behave?
\begin{answer}
TODO
\end{answer}
\end{questions}

\begin{steps}
Compare the results, produced during the last exercises, with each other:

\begin{table}[H]
  \centering
    \begin{tabular*}{0.9\textwidth}{l|l|l|l}
    \toprule
    Metric & SRR022852 & SRR023408 & SRR023408.trimmed \\
    \midrule
    Overall Quality (1-5) & & & \\[0.5\questionspacing]
    \hline
    bp Coverage & & & \\[0.5\questionspacing]
    \hline
    k-mer Coverage & & & \\[0.5\questionspacing]
    \hline
    N50 (k-mer used) & & & \\[0.5\questionspacing]
    \bottomrule
    \end{tabular*}%
  \label{tab:comparison}%
\end{table}%
\begin{answer}
\begin{table}[H]
  \centering
    \begin{tabular*}{0.9\textwidth}{l|l|l|l}
    \toprule
    Metric & SRR022852 & SRR023408 & SRR023408.trimmed \\
    \midrule
    Overall Quality (1-5) & 2 & 5 & 4\\[0.5\questionspacing]
    \hline
    bp Coverage & 136 x (36 bp;11,374,488) & 95x (37bp; 7761796) & 82x (32bp; 7761796)\\[0.5\questionspacing]
    \hline
    k-mer Coverage & 45x & 43x (21); 33x (25) & 30x (21); 20.5x (25)\\[0.5\questionspacing]
    \hline
    N50 (k-mer used) & 68,843 (25) & 2,803 (21) & 2,914 (21)\\[0.5\questionspacing]
    \bottomrule
    \end{tabular*}%
  \label{tab:comparison_result}%
\end{table}%
\end{answer}

\end{steps}

\begin{questions}
What would you consider as the 'best' assembly?
\begin{answer}
SRR022852
\end{answer}

If you found a candidate, why do you consider it as 'best' assembly?
\begin{answer}
Overall data quality and coverage
\end{answer}
\end{questions}