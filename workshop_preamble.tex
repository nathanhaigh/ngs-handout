%
% Start: General Information describing the workshop and the structure of the handouts
%
\newpage

% \section{Do's and Don'ts of the Workshop}
% TODO Add do's and don'ts e.g. email, social media etc 

\section{Providing Feedback}
While we endeavour to deliver a workshop with quality content and documentation
in a venue conducive to an exciting, well run hands-on workshop with a bunch of
knowledgeable and likable trainers, we know there are things we could do better.

Whilst we want to know what didn't quite hit the mark for you, what would be
most helpful and least depressing, would be for you to provide ways to improve
the workshop. i.e. constructive feedback. After all, if we knew something wasn't
going to work, we wouldn't have done it or put it into the workshop in the first
place! Remember, we're experts in the field of bioinformatics not experts in the
field of biology!

Clearly, we also want to know what we did well! This gives us that ``feel good''
factor which will see us through those long days and nights in the lead up to
such hands-on workshops!

With that in mind, we'll provide three really high tech mechanism through which you
can provide anonymous feedback during the workshop:
\begin{enumerate}
  \item A sheet of paper, from a flip-chart, sporting a ``happy'' face and a
  ``not so happy'' face. Armed with a stack of colourful post-it notes, your
  mission is to see how many comments you can stick on the ``happy'' side!
  
  \item Some empty ruled pages at the back of this handout. Use them for your
  own personal notes or for write specific comments/feedback about the workshop
  as it progresses.
  
  \item An online post-workshop evaluation survey. We'll ask you to complete
  this before you leave. If you've used the blank pages at the back of this
  handout to make feedback notes, you'll be able to provide more specific and
  helpful feedback with the least amount of brain-drain!
\end{enumerate}

\section{Document Structure}
We have provided you with an electronic copy of the workshop's hands-on tutorial documents.
We have done this for two reasons: 1) you will have something to take away with you at the 
end of the workshop, and 2) you can save time (mis)typing commands on the command line by using
copy-and-paste.

\emph{We advise you to use Acrobat Reader to view the PDF. This is because it
properly supports some features we have implemented to ensure that
copy-and-paste of commands works as expected. This includes the appropriate
copy-and-paste of special characters like tilde and hyphens as well as skipping
line numbers for easy copy-and-past of whole code blocks.}

\begin{warning}
While you could fly through the hands-on sessions doing
copy-and-paste you will learn more if you take the time, saved from not having to type all those
commands, to understand what each command is doing!
\end{warning}

The commands to enter at a terminal look something like this:
\begin{lstlisting}
tophat --solexa-quals -g 2 --library-type fr-unstranded -j annotation/Danio_rerio.Zv9.66.spliceSites -o tophat/ZV9_2cells genome/ZV9 data/2cells_1.fastq data/2cells_2.fastq
\end{lstlisting}  

The following styled code is not to be entered at a terminal, it is simply to
show you the syntax of the command. You must use your own judgement to
substitute in the correct arguments, options, filenames etc

\begin{lstlisting}[style=command_syntax]
tophat [options]* <index_base> <reads_1> <reads_2>
\end{lstlisting}

The following is an example how of R commands are styled:

\begin{lstlisting}[style=R]
R --no-save
library(plotrix) 
data <- read.table("run_25/stats.txt", header=TRUE) 
weighted.hist(data$short1_cov+data$short2_cov, data$lgth, breaks=0:70)
q()
\end{lstlisting}

The following icons are used in the margin, throughout the documentation to help
you navigate around the document more easily:

% TODO limit the use of some icons throughout as some are clearly overused and confuse the eye
\hspace*{.2cm}\vcent{\includegraphics[height=1cm]{./graphics/info.png}} Important\\
\hspace*{.2cm}\vcent{\includegraphics[height=1cm]{./graphics/notes.png}} For reference\\
\hspace*{.2cm}\vcent{\includegraphics[height=1cm]{./graphics/steps.png}} Follow these steps\\
\hspace*{.2cm}\vcent{\includegraphics[height=1cm]{./graphics/questions.png}} Questions to answer\\
\hspace*{.2cm}\vcent{\includegraphics[height=1cm]{./graphics/warning.png}} Warning - STOP and read\\
\hspace*{.2cm}\vcent{\includegraphics[height=1cm]{./graphics/bonus1.png}} Bonus exercise for fast learners\\
\hspace*{.2cm}\vcent{\includegraphics[height=1cm]{./graphics/bonus2.png}} Advanced exercise for super-fast learners\\

\section{Resources Used}
We have provided you with an environment which contains all the tools and data
you need for the duration of this workshop. However, we also provide details
about the tools and data used by each module at the start of the respective
module documentation.
