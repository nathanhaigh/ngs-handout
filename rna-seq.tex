% Define the top matter
\newcommand{\moduleTitle}{RNA-Seq}
\newcommand{\moduleAuthors}{%
  Myrto Kostadima, EMBL-EBI (kostadmi@ebi.ac.uk)\\
  Remco Loos, EMBL-EBI (remco@ebi.ac.uk)
}
\newcommand{\moduleContributions}{%
  Nathan S. Watson-Haigh (nathan.watson-haigh@awri.com.au)%
}

%
% Start: Module Title Page
% TODO: Define modules as PARTS and define a custom parts page
%


%
% Format the part page as follows
%


%\moduletitlepage{\moduleTitle}{\moduleAuthors}{\moduleContributions}
\chapterstyle{module}
\chapter{\moduleTitle}

%\noindent


\newpage
% End: Module Title Page

\section{Key Learning Outcomes}

TODO

\section{Resources You'll be Using}
Although we have provided you with an environment which contains all the tools and data you will
be using in this module, you may like to know where we have sourced those tools and data from.
 
\subsection{Tools Used}
\begin{description}[style=multiline,labelindent=1.5cm,align=left,leftmargin=2.5cm]
  \item[Tophat] \hfill\\
  	\url{http://tophat.cbcb.umd.edu/}
  \item[Cufflinks] \hfill\\
  	\url{http://cufflinks.cbcb.umd.edu/}
  \item[Samtools] \hfill\\
  	\url{http://samtools.sourceforge.net/}
  \item[BEDTools] \hfill\\
  	\url{http://code.google.com/p/bedtools/}
  \item[UCSC tools] \hfill\\
  	\url{http://hgdownload.cse.ucsc.edu/admin/exe/}
  \item[IGV] \hfill\\
  	\url{http://www.broadinstitute.org/igv/}
  \item[DAVID Functional Analysis] \hfill\\
  	\url{http://david.abcc.ncifcrf.gov/}
\end{description}

\subsection{Sources of Data}
\url{http://www.ebi.ac.uk/ena/data/view/ERR022484}\\
\url{http://www.ebi.ac.uk/ena/data/view/ERR022485}

\newpage

\section{Introduction}
The goal of this hands-on session is to perform some basic tasks in the downstream analysis
of RNA-seq data. We will start from RNA-seq data aligned to the zebrafish genome using Tophat.
We will perform transcriptome reconstruction using Cufflinks and we will compare the gene
expression between two different conditions in order to identify differentially expressed genes.

\section{Prepare the Environment}
We will use a dataset derived from sequencing of mRNA from Danio rerio embryos in two
different developmental stages. Sequencing was performed on the Illumina platform and
generated 76bp paired-end sequence data using polyA selected RNA. Due to the time
constraints of the practical we will only use a subset of the reads.

The data files are contained in the subdirectory called \texttt{data} and are the following:
\begin{description}[style=multiline,labelindent=1.5cm,align=left,leftmargin=2.5cm]
  \item[\texttt{2cells\_1.fastq} and \texttt{2cells\_2.fastq}] \hfill\\
  	These files are based on RNA-seq data of a 2-cell zebrafish embryo
  \item[\texttt{6h\_1.fastq} and \texttt{6h\_2.fastq}] \hfill\\
  	These files are based on RNA-seq data of zebrafish embryos 6h post fertilization
\end{description}

\begin{steps}
Open the Terminal and go to the directory where the data is stored:
\begin{lstlisting}
   cd ~/RNA-seq/
\end{lstlisting}
\end{steps}

%\reversemarginpar\marginpar{\vskip+0em\hfill\includegraphics[height=1cm]{./graphics/warning.png}}
%\textcolor{red}{
\begin{warning}
  All commands entered into the terminal for this tutorial should be from within the
  \texttt{RNA-seq} directory above.
\end{warning}

\begin{steps}
Check that the \texttt{data} directory contains the above-mentioned files by typing:
\begin{lstlisting}
   ls -l data
\end{lstlisting}
\end{steps}

\section{Alignment}
There are numerous tools performing short read alignment and the choice of aligner
should be carefully made according to the analysis goals/requirements. Here we will
use Tophat, a widely used ultrafast aligner that performs spliced alignments.

Tophat is based on Bowtie to perform alignments and uses an indexed genome for the
alignment to keep its memory footprint small. We have already seen how to index the
genome (see Alignment hands-on session), therefore for time purposes we have already
generated the index for the zebrafish genome and placed it under the \texttt{genome}
subdirectory.

\begin{steps}
Tophat has a number of parameters in order to perform the alignment. To view them all type:
\begin{lstlisting}
   tophat --help
\end{lstlisting}
\end{steps}

\begin{information}
The general format of the tophat command is:
\begin{lstlisting}
   tophat [options]* <index_base> <reads_1> <reads_2>
\end{lstlisting}

Where the last two arguments are the \texttt{.fastq} files of the paired end reads, and the argument
before is the basename of the indexed genome.
\end{information}

%\reversemarginpar\marginpar{\vskip+0em\hfill\includegraphics[height=1cm]{./graphics/notes.png}}
\begin{note}
Like with Bowtie before you run Tophat, you have to know which quality encoding the fastq
formatted reads are in.
\end{note}

\begin{questions}
Can you tell which quality encoding our fastq formatted reads are in?

\vspace*{2cm}

HINT: Look at the first few reads in the file \texttt{data/2cells\_1.fastq} and
then compare the quality strings with the table found at:
\url{http://en.wikipedia.org/wiki/FASTQ_format#Encoding}. 
\begin{lstlisting}
  head -n 20 data/2cells_1.fastq
\end{lstlisting}
\end{questions}

%\reversemarginpar\marginpar{\vskip+0em\hfill\includegraphics[height=1cm]{./graphics/notes.png}}
\begin{note}
Some other parameters that we are going to use to run Tophat are listed below:
\begin{description}[style=multiline,labelindent=0cm,align=right,leftmargin=5cm,font=\ttfamily]
 \item[-g] Maximum number of multihits allowed. Short reads are likely to map to
 more than one locations in the genome even though these reads can have originated
 from only one of these regions. In RNA-seq we allow for a restricted number of
 multihits, and in this case we ask Tophat to report only reads that map at most
 onto 2 different loci.
 \item[--library-type] Before performing any type of RNA-seq analysis you need
 to know a few things about the library preparation. Was it done using a
 strand-specific protocol or not? If yes, which strand? In our data the protocol
 was NOT strand specific.
 \item[-J] Improve spliced alignment by providing Tophat with annotated splice
 junctions. Pre-existing genome annotation is an advantage when analysing RNA-seq
 data. This file contains the coordinates of annotated splice junctions from Ensembl.
 These are stored under the sub-directory \texttt{annotation} in a file called
 \texttt{ZV9.spliceSites}.
 \item[-o] This specifies in which subdirectory Tophat should save the output
 files. Given that for every run the name of the output files is the same, we
 specify different directories for each run.
\end{description}
\end{note}

Since it takes a long time to perform spliced alignments, even for this subset of
reads, we have pre-aligned the \texttt{2cells} data for you using the following command:
\begin{warning}
  DO NOT run this command yourself - you may overwrite the output files of the
  pre-aligned data.
\end{warning}

\begin{lstlisting}
  tophat --solexa-quals -g 2 --library-type fr-unstranded -j annotation/Danio_rerio.Zv9.66.spliceSites -o tophat/ZV9_2cells genome/ZV9 data/2cells_1.fastq data/2cells_2.fastq
\end{lstlisting}

\begin{questions}
Using the above command as a guide, can you
devise and run the command for aligning the \texttt{6h} data?

\vspace*{5cm}

\begin{warning}
The alignment will take approximately 17 minutes. In the meantime please move
on with the practical and we will get back to the terminal once the alignment
is done.
\end{warning}

\end{questions}

\begin{steps}
We will firstly look at some of the files produced by Tophat. For this please
open the \texttt{RNA-seq} directory which can be found on your Desktop. Click on the \texttt{tophat}
subdirectory and then on the directory called \texttt{ZV9\_2cells}.

Tophat reports the alignments in a BAM file called \texttt{accepted\_hits.bam}.
Among others it also creates a junctions.bed files that stores the coordinates
of the splice junctions present in your dataset, as these have been extracted
from the spliced alignments.

Now we will load the BAM file and the splice junctions onto IGV to visualize
the alignments reported by Tophat.

In order to launch IGV type double click on the IGV 2.1 icon on your Desktop.
Ignore any warnings and when it opens you have to load the genome of interest.

On the top left of your screen choose from the drop down menu Zebrafish (Zv9).
Then in order to load the desire files go to:

\begin{lstlisting}
  File >> Load from File
\end{lstlisting}

On the pop up window navigate to \texttt{Desktop} \textgreater\textgreater\ \texttt{RNA-seq}
\textgreater\textgreater\ \texttt{tophat} \textgreater\textgreater\ \texttt{ZV9\_2cells} directory
and select the file \texttt{accepted\_hits.sorted.bam}. Once the file is loaded
right-click on the name of the track on the left and choose Rename Track.
Give the track a meaningful name.

Follow the same steps in order to load the \texttt{junctions.bed} file from the same
directory.

Finally following the same process load the Ensembl annotation
\texttt{Danio\_rerio.Zv9.66.gtf} stored under directory \texttt{annotation} under the
\texttt{RNA-seq} directory.

On the top middle box you can specify the region you want your browser to
zoom. Type: \texttt{chr12:20,270,921-20,300,943}.
\end{steps}

\begin{questions}
Can you identify the splice junctions from the BAM file?

\vspace{2cm}

Are the junctions annotated for \texttt{CBY1} consistent with the annotation?

\vspace{2cm}

Are all annotated genes, from both RefSeq and Ensembl, expressed?

\vspace{2cm}

\end{questions}

\begin{information}
We already know that in order to load a BAM file onto IGV we need to have
this file sorted by genomic location and indexed. Here�s a reminder of the
command syntax to perform these tasks

\begin{lstlisting}
  samtools sort [bam_file_to_be_sorted] [prefix_of_sorted_bam_output_file]
  samtools index [sorted_bam_file]
\end{lstlisting}

\end{information}

\begin{steps}
Once Tophat finishes running, sort the output BAM file and then index the
sorted BAM file using the information above to guide you.

Then load them onto IGV following the steps above.
\end{steps}

\newpage
\section{Isoform Expression and Transcriptome Assembly}
There are a number of tools that perform reconstruction of the transcriptome
and for this workshop we are going to use Cufflinks. Cufflinks can do
transcriptome assembly either ab initio or using a reference annotation. It
also quantifies the isoform expression in FPKMs.

A reminder from the presentation this morning that FPKM stands for `Fragments
Per Kilobase of exon per Million fragments mapped'. 

\begin{steps}
Cufflinks has a number of parameters in order to perform transcriptome
assembly and quantification. To view them all type:

\begin{lstlisting}
  cufflinks --help
\end{lstlisting}
\end{steps}

We aim to reconstruct the transcriptome for both samples by using the Ensembl
annotation both strictly and as a guide. In the first case Cufflinks will only
report isoforms that are included in the annotation, while in the latter case
it will report novel isoforms as well.

The annotation from Ensembl of Danio rerio is stored under the directory
\texttt{annotation} in a file called \texttt{Danio\_rerio.Zv9.66.gtf}.

\begin{information}
The general format of the cufflinks command is:
\begin{lstlisting}
 cufflinks [options]* <aligned_reads.(sam|bam)>
\end{lstlisting}
Where the input is the aligned reads (either in SAM or BAM format).
\end{information}

\begin{note}
Some of available parameters of Cufflinks that we are going to use to run
Cufflinks are listed below:
\begin{description}[style=multiline,labelindent=0cm,align=right,leftmargin=5cm,font=\ttfamily]
  \item[-o] Output directory
  \item[-G] Tells Cufflinks to use the supplied annotation strictly in order
  to estimate isoform annotation
  \item[-b] Instructs Cufflinks to run a bias detection and correction algorithm
  which can significantly improve accuracy of transcript abundance estimates.
  To do this Cufflinks requires a multi-fasta file with the genomic sequences
  against which we have aligned the reads
  \item[-u] Tells Cufflinks to do an initial estimation procedure to more
  accurately weight reads mapping to multiple locations in the genome
  (multi-hits). 
  \item[--library-type] See Tophat parameters
\end{description}
\end{note}


\begin{steps}
In the terminal type:
\begin{lstlisting}
  cufflinks -o cufflinks/ZV9_2cells_gff -G annotation/Danio_rerio.Zv9.66.gtf -b genome/Danio_rerio.Zv9.66.dna.fa -u --library-type fr-unstranded tophat/ZV9_2cells/accepted_hits.bam
\end{lstlisting}
\end{steps}

\begin{warning}
  Don't forget to change the output directory for the question section below. Otherwise your command will
  overwrite the results of the previous run.
\end{warning}

\begin{questions}
Using the above command, which we used to pre-align the \texttt{2cells} data, can you devise and run a
command for the \texttt{6h} data?

\vspace{5cm}

Take a look at the output directories that have been created. The results from
Cufflinks are stored in 4 different files, see below.
\end{questions}

\begin{information}
Here's a short description of these files:

\begin{description}[style=multiline,labelindent=1.5cm,align=left,leftmargin=2.5cm,font=\ttfamily]
  \item[genes.fpkm\_tracking] \hfill\\
  	Contains the estimated gene-level expression values.
  \item[isoforms.fpkm\_tracking] \hfill\\
  	Contains the estimated isoform-level expression values.
  \item[skipped.gtf] \hfill\\
    No description  % TODO: fill in description
  \item[transcripts.gtf] \hfill\\
  	This GTF file contains Cufflinks' assembled isoforms.
\end{description}

The complete documentation can be found at:
\url{http://cufflinks.cbcb.umd.edu/manual.html#cufflinks_output}
\end{information}

Now in order to perform guided transcriptome assembly (transcriptome assembly
that reports novel transcripts as well) we need to change the \texttt{�G}
option of the previous command to \texttt{�g} instead.
This tells Cufflinks to assemble the transcriptome using the supplied
annotation as a guide, thus allowing for novel transcripts.

\begin{information}
Performing the guided transcriptome analysis for the \texttt{2cells} and
\texttt{6h} data sets would take 15-20min each. Therefore, we have
pre-computed these for you and have the results available under subdirectories:
\texttt{cufflinks/ZV9\_2cells} and \texttt{cufflinks/ZV9\_6h}.
\end{information}

\begin{questions}
Rather than running the commands for guided transcriptome assembly, which would
take too much time, please have a think about the cufflinks command you would
need to use to run in order to perform a guided transcriptome assembly
for the \texttt{2cells} dataset. Write your answer below:

\vspace{5cm}

\end{questions}

\begin{steps}
Go back to the IGV browser and load the file transcripts.gtf which is located
in the subdirectory \texttt{cufflinks/ZV9\_2cells/}. Rename the track into
something meaningful.

This file contains the transcripts that Cufflinks assembled based on the
alignment of our reads onto the genome.
\end{steps}

\begin{questions}
In the search box type \texttt{ENSDART00000082297} in order for the browser to zoom in
to the gene of interest. Compare between the already annotated transcripts and
the ones assembled by Cufflinks. Do you observe any difference?

\vspace{3cm}

\end{questions}

\section{Differential Expression}
One of the stand-alone tools that perform differential expression analysis is
Cuffdiff. We use this tool to compare between two conditions; for example
different conditions could be control and disease, or wild-type and mutant, or
various developmental stages. In our case we want to identify genes that are
differentially expressed between two developmental stages; a \texttt{2cell}
embryo and \texttt{6h} post fertilization.

\begin{information}
The general format of the cuffdiff command is:
\begin{lstlisting}
  cuffdiff [options]* <transcripts.gtf> <sample1_replicate1.sam[,...,sample1_replicateM]> <sample2_replicate1.sam[,...,sample2_replicateM.sam]>
\end{lstlisting}

Where the input includes a \texttt{transcripts.gtf} file, which is an annotation
file of the genome of interest, and the aligned reads (either in SAM or BAM
format) for the conditions.
Some of the Cufflinks options that we will use to run the program are:
\begin{description}[style=multiline,labelindent=0cm,align=right,leftmargin=5cm,font=\ttfamily]
  \item[-o] output directory
  \item[-L] labels for the different conditions
  \item[-T] Tells Cuffdiff that the reads are from a time series experiment
  \item[-b] Same as for Cufflinks
  \item[-u] Same as for Cufflinks
  \item[--library-type] Same as for Cufflinks
\end{description}
\end{information}

\begin{steps}
To run cufdiff type on the terminal:
\begin{lstlisting}
  cuffdiff -o cuffdiff/ -L ZV9_2cells,ZV9_6h -T -b genome/Danio_rerio.Zv9.66.dna.fa -u --library-type fr-unstranded annotation/Danio_rerio.Zv9.66.gtf tophat/ZV9_2cells/accepted_hits.bam tophat/ZV9_6h/accepted_hits.bam
\end{lstlisting}
\end{steps}

\begin{warning}
In the command above we have assumed that the directory where you stored the
results of Tophat for dataset \texttt{6h} was named \texttt{ZV9\_6h}. If this
is not the case please change the previous command accordingly otherwise you
will get an error.
\end{warning}

\begin{steps}
We are interested in the differential expression at the gene level. The results
are reported by Cuffdiff in the file \texttt{cuffdiff/gene\_exp.diff}. 
Look at the first few lines of the file using the following command:
\begin{lstlisting}
  head -n 20 cuffdiff/gene_exp.diff
\end{lstlisting}

We would like to see which are the most significantly differentially expressed
genes. Therefore we will sort the above file according to the q value
(corrected p value for multiple testing). The result will be stored in a
different file called \texttt{gene\_exp\_qval.sorted.diff}.
\begin{lstlisting}
  sort -t$'\t' -g -k 13 cuffdiff/gene_exp.diff > cuffdiff/gene_exp_qval.sorted.diff
\end{lstlisting}

Look again at the first few lines of the sorted file by typing:
\begin{lstlisting}
  head -n 20 cuffdiff/gene_exp_qval.sorted.diff
\end{lstlisting}

Copy the Ensembl identifier of one of these genes. Now go back to the IGV
browser and paste it in the search box. Look at the raw aligned data for the
two datasets.
\end{steps}

\begin{questions}
Do you see any difference in the gene coverage between the two conditions that
would justify that this gene has been called as differentially expressed?

\vspace{3cm}

\end{questions}

\begin{warning}
Note that the coverage on the Ensembl browser is based on raw reads and no
normalisation has taken place contrary to the FPKM values.
\end{warning}

\begin{bonus}
\section{Functional Annotation of Differentially Expressed Genes}
After you have performed the differential expression analysis you are interested
in identifying if there is any functionality enrichment for your differentially
expressed genes.
On your Desktop click:
\begin{lstlisting}
  Applications >> Internet >> Firefox Web Browser
\end{lstlisting}
And go to the following URL: \url{http://david.abcc.ncifcrf.gov/}
On the left side click on Functional Annotation. Then click on the Upload tab.
Under the section Choose from File, click Choose File and navigate to the
\texttt{cuffdiff} directory. Select the file called \texttt{globalDiffExprs\_Genes\_qval.01\_top100.tab}.
Under Step 2 select ENSEMBL\_GENE\_ID from the drop-down menu. Finally select
Gene list and then press Submit List.
Click on Gene Ontology and then click on the CHART button of the GOTERM\_BP\_ALL item.

\begin{questions}
Do these categories make sense given the samples we�re studying?

\vspace{3cm}

Browse around DAVID website and check what other information are available.
\end{questions}
\end{bonus}

\newpage

\section{References}
\begin{enumerate}
  \item Trapnell, C., Pachter, L. \& Salzberg, S. L. TopHat: discovering splice
  junctions with RNA-Seq. Bioinformatics 25, 1105�1111 (2009).
  \item Trapnell, C. et al. Transcript assembly and quantification by RNA-Seq
  reveals unannotated transcripts and isoform switching during cell
  differentiation. Nat. Biotechnol. 28, 511�515 (2010).
  \item Langmead, B., Trapnell, C., Pop, M. \& Salzberg, S. L. Ultrafast and
  memory-efficient alignment of short DNA sequences to the human genome.
  Genome Biol. 10, R25 (2009).
  \item Roberts, A., Pimentel, H., Trapnell, C. \& Pachter, L. Identification
  of novel transcripts in annotated genomes using RNA-Seq. Bioinformatics 27,
  2325�2329 (2011).
  \item Roberts, A., Trapnell, C., Donaghey, J., Rinn, J. L. \& Pachter, L.
  Improving RNA-Seq expression estimates by correcting for fragment bias.
  Genome Biol. 12, R22 (2011).
\end{enumerate}

\chapterstyle{workshop}